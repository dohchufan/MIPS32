\documentclass{ctexart}
\usepackage{amsmath}
\usepackage{geometry}
\usepackage{color}
\usepackage{diagbox}
\usepackage{lmodern}
\usepackage{listings}
\usepackage{fontspec}
\usepackage{graphicx}
\usepackage{longtable}
\usepackage[cache=false]{minted}
\usepackage[titletoc]{appendix}
\usepackage{lipsum}
\usepackage[colorlinks,linkcolor = black]{hyperref}

\newcommand{\blue}{\textcolor{blue}}
\newcommand{\rotate}{\rotatebox{90}}


\definecolor{mygreen}{rgb}{0,0.6,0}
\definecolor{mygray}{rgb}{0.5,0.5,0.5}
\definecolor{mymauve}{rgb}{0.58,0,0.82}
\lstset{ %
backgroundcolor=\color{white},   % choose the background color
basicstyle=\ttfamily,        % size of fonts used for the code
columns=fullflexible,
breaklines=true,                 % automatic line breaking only at whitespace
captionpos=b,                    % sets the caption-position to bottom
tabsize=4,
commentstyle=\color{mygreen},    % comment style
escapeinside={\%*}{*)},          % if you want to add LaTeX within your code
keywordstyle=\color{blue},       % keyword style
stringstyle=\color{mymauve}\ttfamily,     % string literal style
frame=single,
% rulesepcolor=\color{red!20!green!20!blue!20},
% identifierstyle=\color{red},
% language=c++,
}

\geometry{left = 3cm,right = 3cm,top = 2.5cm, right = 2.5cm}

\begin{document}
	\title{数字逻辑与处理器基础实验 \\ 32位MIPS处理器设计}
	\author{孙伟艺\thanks{无52 2015011010}\\ 白钦博\thanks{无52 2015010996} \\ 王敏虎 \thanks{无52 2015011003}}
	\date{\today}
	\maketitle
	\clearpage

	\section{实验目的}
	\begin{itemize}
		\item 熟悉现代处理器的基本工作原理
		\item 掌握单周期和流水线处理器的设计方法
	\end{itemize}

	\section{设计方案}
		\subsection{ALU}
			(孙伟艺)
		\subsection{单周期数据通路}
			(孙伟艺)
		\subsection{流水线数据通路}
			(白钦博)
		\subsection{外设}
		本次实验需要使用LED灯、七段数码管、串口等外设参与CPU工作,为CPU提供运算所需要的数据并将CPU的运算结果表示出来。外设不是CPU的硬件组成部分,
		对于CPU,外设被看作内存中的一个普通地址,CPU不需要了解外设工作的具体细节,只需要执行程序员编写好的程序,将数据存入对应的地址即可,后续工作
		应当由外设电路独立完成。

		本次实验中,地址\verb"0x40000000"到\verb"0x40000020"被用于外设地址。在CPU的连接中,这些地址被连接到专门的电路而非内存中。

		地址如下表被分配给外设。

		\begin{table}[ht]
			\centering
			\begin{tabular}{|c|c|}
				\hline
				地址 & 功能  \\
				\hline
				\verb"0x40000000" & 定时器TH \\
				\verb"0x40000004" & 定时器TL \\
				\verb"0x40000008" & 定时器控制TCON \\
				\verb"0x4000000C" & LED \\
				\verb"0x40000010" & Switch \\
				\verb"0x40000014" & 七段数码管 \\
				\verb"0x40000018" & UART发送数据 \\
				\verb"0x4000001C" & UART接收数据 \\
				\verb"0x40000020" & 串口状态 \\ 
				\hline
			\end{tabular}
		\end{table}


以下为实验要求编写的各外设的说明

			\subsubsection{LED灯}
			LED灯是本次实验中比较简单的外设。在CPU访问\verb"0x4000000C"地址时将对应位数赋给相应管脚即可。
			\subsubsection{七段数码管}
			本次实验使用四个七段数码管,由于要求使用软件译码,硬件部分较为简单,七段数码管使用12位进行控制,前4位表示当前点亮的数码管位置,
			后八位表示七段数码管各管脚的电平,与LED灯类似的是,当CPU访问\verb"0x40000014"时,将对应位数赋给对应接线即可。复杂的译码和控制部分
			将在汇编程序中进行。
			\subsubsection{switch开关}
			switch开关是输入设备,不能被写入,当CPU试图读\verb"0x40000010"时,将对应连线上的电平返回。
			\subsubsection{串口}
			
			串口使用轮询方式编写,三位\verb"0x40000018",\verb"0x4000001C",\verb"0x40000020"控制,其中\verb"0x40000020"是串口状态位,
			其最后两位中的第一位用来标示是否收到新的数据,第二位用来表示目前串口的发送状态。当数据被写入\verb"0x40000018"时,串口将
			自动发送其后八位并修改串口发送状态,当\verb"0x4000001C"访问后,串口将修改串口状态位,将接收标志位置为低以等待下一个数据。

			编写汇编程序时,应当首先访问串口状态位,确定串口已经接收到数据,再访问串口接收数据地址。使用轮询方式的一个问题即是,如果轮询时间过长,串口中的数据
			可能会丢失,但是在本实验的条件下,9600波特率的串口接收一次数据的时间足以CPU完成上万个时钟周期的运算,可以认为串口数据能够被及时访问。


		\subsection{汇编程序与汇编器}
		\subsubsection{汇编程序}
			本次实验中的汇编程序由两部分组成,一部分是从串口读入数据并运行算法计算最大公约数,另一部分是数码管的译码和显示,
			由于实验中定时器的中断只用来完成数码管的扫描,因此数码管的译码和显示代码就是中断处理代码。

			第一部分代码如下。首先启动定时器,而后检查串口标志位,当串口标志位有效时读串口数据。待输入的两个数据均读取完成后,
			计算结果并访问外设以显示结果。当完成任务后,CPU将进入无限循环的状态以便观察结果。

			第二部分的代码如首先将处理与定时器相关的中断,而后保护现场,由于程序比较简单,主程序与中断处理程序未使用相同的寄存器,
			因此这一步在代码中未体现。中断处理代码首先检查数码管对应外设数据,并移动扫描位,而后进行软件译码,软件译码的过程即case块
			语法转换成汇编语法,较为繁琐,因此在报告中删去部分。软件译码完成后,修改数码管外设对应地址值,重新启动定时器,回到主程序继续执行。

			两部分代码见于附录。


		\subsubsection{汇编器}
			汇编器使用python语言编写。汇编器依照以下步骤执行工作:
			\begin{enumerate}
				\item 遍历代码,计算Label名称对应的地址值
				\item 遍历代码,将Label名称表示的地址转换成相对地址或绝对地址
				\item 遍历代码,将汇编程序转换成机器码
			\end{enumerate}
			匹配和替换工作主要使用正则表达式完成,主要代码见于附录。

	\section{关键代码与文件清单}
\begin{verbatim}
├─assemble
│      assemble.py  汇编器程序
│      data.txt		实验使用测试程序
│      encode.py	机器码转换为verilog文件结构程序
│      m_code.txt	汇编器转换机器码文件
│      verilog.txt	直接贴入verilog rom.v 文件
│
├─OneCycle
│  │  Adder.v			全加器
│  │  ALU.v				单周期ALU
│  │  Control.v			单周期控制信号生成文件
│  │  CPU.v				单周期CPU结构文件
│  │  DataMem.v			单周期Data Memory
│  │  digitube_scan.v	
│  │  divclk.v			分频模块
│  │  Peripheral.v		外设模块
│  │  regfile.v			寄存器
│  │  rom.v				指令存储器
│  │  UART.v			串口相关电路实现
│  │
│  └─output_files
├─Pipeline
│  │  Adder.v			全加器
│  │  ALU.v				流水线ALU
│  │  Control.v			流水线控制信号生成文件
│  │  CPU.v				流水线CPU结构文件
│  │  DataMem.v			流水线Data Memory
│  │  digitube_scan.v	
│  │  Peripheral.v		外设模块
│  │  regfile.v			寄存器
│  │  rom.v				指令存储器
│  │  UART.v			串口相关电路实现
│  │
│  └─output_files
\end{verbatim}
	\section{仿真结果与分析}
	\section{综合情况}
	单周期各项性能指标如下:
	\begin{center}
		\begin{tabular}{|c|c|}
		\hline
		性能指标 & \\
		\hline
		Total Logic Elements & 4498(14\%) \\
		Total Registers & 3772(11\%) \\
		Total Pins & 48(10\%) \\
		Restricted Fmax & $41.55 MHz$ \\  
		\hline
		\end{tabular}
	\end{center}
	
	流水线各项性能指标如下:
	\begin{center}
		\begin{tabular}{|c|c|}
		\hline
		性能指标 & \\
		\hline
		Total Logic Elements & 3803(11\%) \\
		Total Registers & 3695(11\%) \\
		Total Pins & 48(10\%) \\
		Restricted Fmax & $88.0 MHz$ \\  
		\hline
		\end{tabular}
	\end{center}
	\section{硬件调试情况}
	\section{思想体会}

\end{document}